
In 2003, Blei, Ng and Jordan \cite{lda_vem} presented the Latent
Dirichlet Allocation (LDA) model and a Variational
Expectation-Maximization algorithm for training the model.  In 2004,
Griffiths and Steyvers \cite{lda_gibbs} derived a Gibbs sampling
algorithm for learning LDA.  Since then, Gibbs sampling was shown more
efficient than other LDA training algorithms including variational EM
and Expectation-Propagation \cite{ep-lda}.  This efficiency is due to
a instrisic property of LDA -- the conjugacy between the Dirichlet
prior the multinomial likelihood.  For this reason, Gibbs sampling
algorithms were derived for inference in many models that extends LDA
\cite{tot} \cite{rart} \cite{hlda} \cite{slda} \cite{pam}.

To further improve the efficiency of the Gibbs sampling algorithm for
LDA, researchers tried to distribute the computation on multiple
computers \cite{dist-lda-gibbs} or to optimize the Gibbs sampling
speed on each computer \cite{fastlda}.

Since November 2007, I started to work on developing distributed
computing solutions of topic models.  Industrial solutions are often
required to train models using massive data sets, so I need to express
training algorithms using modern distributed computing models, in
particular, MapReduce, BSP and MPI.

This document is my learning and working note.  If you are interested
with large scale computing of latent topic models, I hope this
document could be helpful in the first stage of your work.

%%% Local Variables: 
%%% mode: latex
%%% TeX-master: "llt"
%%% End: 
